\title{Un document latex}
\author{\textit{Polytech}}
\date{}

\begin{document}
\maketitle

\section{Document \textit{Latex}}

Un document \textbf{Latex} est un document texte qui contient certaines
commandes qui seront traitées par un compilateur. \\
L'une des forces de Latex est de pouvoir générer des documents très 
propres, \textit{très structurés} avec une numérotation automatique.

\subsection{Fonctionnement}

En résumé, Latex :

\begin{itemize}
\item permet d'écrire des documents à l'aide d'un simple éditeur de texte,
\item s'occupe de la mise en page,
\item offre un résultat \textit{incomparable}. 
\item La réalisation d'un document se fait en 3 phases :
    \begin{enumerate}
\item Edition du texte du document (vi, kwrite, xemacs, \ldots)
    \item Compilation du texte (latex, pdflatex)
    \item Visualisation du document généré (evince, acroread)
    \end{enumerate}
\end{itemize}

\end{document}
